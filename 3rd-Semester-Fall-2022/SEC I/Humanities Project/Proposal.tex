\documentclass[man]{apa7}

\usepackage{natbib}
\usepackage{float}
\usepackage{hyperref}

% Title Page
\title{Clean Fuel at Rowan University}
\author{Aidan Sharpe, Kevin Hack, Tyler Torres, Rahaa Kumeresan}
\authorsaffiliations{Rowan University}
\shorttitle{}


\abstract{Personal transportation is a necessity on any university campus. However, as we become more aware of the environmental harm that fuels like gasoline and diesel cause, it is important to weigh alternative solutions. Whether it be due to price, inaccessible charging, safety concerns, or simply the limited practicality that comes with having to take long charging breaks, electric vehicles are not an option for many. Furthermore, since hydrogen technology is not yet mature, the simplest way to incorporate a green fuel solution is by changing the source of our fuels. In other words, the best solution is to create fuels that are chemically similar to traditional combustion fuels through a renewable, carbon-zero process.}


\begin{document}
\maketitle


\section{INTRODUCTION}
	Next time you are on campus, take a walk outside. Take the time to notice the trees, bushes, and creeks. Rowan University has a quite bio-diverse campus, but you might notice something other than nature on your walk. Throughout the day, every sidewalk and path is full of students walking, biking, and skateboarding.
	\\
	However, when classes are over and the evening sets in, the activity on these paths slows to a trickle. Yet this lack of foot traffic should not, and does not indicate a lack of campus-wide activity. For many, the night has just begun. Some may go on drives, some may go to parties, commuters and faculty make the trek back home, and some stay in to hang out or do homework.
	\\
	In general, there exists a transition from localized movement to regional trips at this time. As the scope of travel increases, so does the demand for automotive transport. Therefore, after classes end, more fuel is burned and, in turn, more air pollution is produced.
	\\
	After walking for a little while, you may have noticed the unmistakable Rowan University Shuttle Bus system. Whether it is too cold, stormy, or it is just a long walk, students can opt to take the shuttle around campus. The only problem is that these buses trade personal convenience for environmental sustainability. Running shuttle buses 18 hours per day at 5 days a week will inevitably result in large amounts of pollution, regardless of how full each bus is. Therefore, this pollution is solely dependent on fuel source and not transportation demand.
	\\
	Simply put, Rowan University pollutes---a lot. There are several possible solutions: anything from making the switch to electric vehicles to cutting down on transport altogether. However, considering alternative fuel sources on campus will allow for emissions to be offset without massively disruptive side-effects.

\section{METHODS}
	To understand the variables at play, a three-phase research process was conducted. The first step involved understanding the preexisting literature. Stage two surveyed a representative population of Rowan University students and staff. Finally, stage three required field research to access lesser known information.
	\\
	We first broke ground by reading preexisting literature on automotive pollution. It quickly became apparent that battery electric vehicles (BEVs) were not at all the savior that many make them out to be. From cradle to grave, BEVs came with a plethora of environmental and social issues. Their lithium-ion batteries are non-recyclable and extracting the lithium from brines often comes with unwanted ecological risks \citep{Flexer2018}. Once the BEV is produced, however, it almost always performs better than comparable internal combustion engine vehicles (ICEVs) in terms of emissions \citep{Franzo2020}. Once again, however, since the lithium cannot be recycled, the end-of-life phase for BEV batteries is almost inevitably electronic waste \citep{Flexer2018}.
	\\
	In addition to environmental problems, it was also apparent that a series of social issues arose. For example, according to German car manufacturer, Mercedes-Benz, they are planning on offering an \$1,200 per month "Acceleration Increase" subscription package on their electric Mercedes-EQE and Mercedes-EQS vehicles \citep{Mercedes2022}. It is unacceptable to be expected to pay such a high recurring charge to simply change a number in software without any human input whatsoever. This is just one example of how easy it is to make unethical business decisions through software.
	\\
	Therefore, not only would it be cheaper to avoid an enforced transition to electric vehicles, but social and environmental hazards can also be dodged. For this reason, research was also conducted to explore alternative "drop-in" replacements for traditional fuels.
	\\
	One such replacement is synthetic fuel. Due to the molecular compositions of gasoline and diesel, it is possible to generate them from nothing but water and carbon-dioxide \citep{Simakov2017}. This is not simply theoretical chemistry either. German car-maker, Audi, in partnership with German renewable energy company, Sunfire, have demonstrated a process to create synthetic diesel fuel \citep{Ferris2017}.
	\\
	Another such replacement, and one that could possibly solve even more problems for Rowan University, is bio-fuel. Since Rowan University produces a surplus of food waste, it would be quite helpful if some of it could be put to use. 
	\\
	After doing preliminary research, it was important to understand how the current literature applied to Rowan University. The first step in doing so required surveying students, staff, and faculty to gauge various attributes pertaining to vehicle use. The survey was sent throughout the Henry M. Rowan College of Engineering and was eventually completed by 136 people. Questions regarded demographics, personal vehicle make, model and year, shuttle service usage, and personal vehicle usage. The data collected was then analyzed and compiled to generate more meaningful figures.
	\\
	The final stage of research was the most hands-on. Since there are no publicly-available figures on Rowan University's shuttles in terms of gas mileage and distance traveled, a shuttle driver was interviewed. Similarly, the university does not publicly post food waste statistics. For this reason, Gourmet Dining, the university food program was contacted. 
	\\
	Finally, since solar power is proportional to the surface area of solar panels, several on-campus parking lots were measured to calculate surface area. In doing so, the amount of power that could be generated from solar parking lot coverings could be calculated.
	
	\section{RESULTS}
	After surveying 136 people, 13 submissions were invalidated due to improper question response. Of the remaining 123 responses, 77 were undergraduate students, 20 were graduate students, and 26 were faculty or staff. 
	\\
	All respondents were asked to provide information about transportation habits. As seen in Table \ref{tbl:MilesPerWeek}, commuters are responsible for the majority of miles driven. While this may not come as any surprise, it should be noted that a decent portion of these commuters are faculty and staff. It just goes to show that there are a lot of miles that cannot be eliminated.
	\\
	\begin{table}[b]
		\afterpage{\clearpage}
		\caption{Miles Driven Per Week by Demographic}
		\begin{tabular}{|l|l|}
			\hline
			\textbf{Demographic} & \textbf{Miles Per Week} \\ \hline
			On-Campus All        & 7.7                     \\ \hline
			On-Campus Drivers    & 29.3                    \\ \hline
			Commuters            & 150.4                   \\ \hline
			All					 & 101.7				   \\ \hline
		\end{tabular}
		\label{tbl:MilesPerWeek}
	\end{table}
	In addition to driving habits, vehicle type was also surveyed. After researching each automobile based on year, make, and model, it was determined that the average fuel economy at Rowan University is roughly 29.7 miles per gallon. Therefore, an average person driving the average car at Rowan would use roughly 3.42 gallons of gasoline per week. However, when accounting for which cars are being used for which distances, the actual per capita fuel usage is about 4 gallons per week.
	\\
	The final statistic surveyed was shuttle service usage. Polls showed that only about 10\% of students at Rowan University have ever used the shuttle. Field research yielded similar results. The shuttles were hardly filling any seats. Additionally, after interviewing a shuttle driver, it was revealed that bus routes were anywhere from 65 to 70 miles per day. The driver was also able to assist in determining the fuel economy of the shuttles. Since the buses are modified Ford E-350s, there are no public statistics. Dividing the claimed 135 miles since fill up by the displayed 28 gallons used yielded a fuel economy of about 4.8 miles per gallon.
	\\
	In addition to reaching out to shuttle bus drivers, one of the heads of \textit{Gourmet Dining} staff, Albert Irons, was contacted. He responded, "Internally [Gourmet Dining works] to mitigate this waste through our "Waste Not" program.  In the program we look at waste and try to determine the root cause." Although there are mitigation tactics in place, there will always be waste. Therefore, handling it in such a way that causes minimal environmental damage is very important.
	\\
	Once the biomass for the bio-fuel was sourced, attention was turned towards the energy required to convert it into bio-fuel. According to \textit{Energy and cost analyses of biodiesel production from waste cooking oil}, bio-diesel requires 30 megajoules of energy to produce one liter \citep{Ahmad2014}. That is a lot of energy. So research ended with a question: where to get enough energy.
	\section{CONCLUSIONS}
	The final portion of research involved measuring parking lots. At this point a solution began to come together. So why parking lots? One of the many advantages of parking garages is the cover that they provide. Whether it be rain cover on a stormy day or offering shade in the sweltering summer heat, having some sort of roof makes all the difference. One nice thing about parking lots, however, is their immense surface area. Other than parking, though, that surface area goes more or less to waste. Understanding the advantages of a cover, there would be many advantages putting a solar roof on several of Rowan University's parking lots.
	\\
	Tesla advertises their solar panels to be 400 watts at 3,065 in\textsuperscript{2}. That is almost exactly two square meters. Surface area measurements of several Rowan University parking lots are seen in Table \ref{tbl:LotAreas}. At 400 watts for two square meters, Tesla solar panels can supply 200 watts per square meter. Assuming the university invested in all of the listed parking lots, expected power output could reach 14.4 megawatts or 14.4 megajoules of energy per second. That is enough power to make a liter of biodiesel in a little over two seconds!
	\\
	Therefore, by re-purposing food waste and generating power with quality-of-life enhancing parking lot covers, it would be possible to replace a large portion of Rowan University's fuel usage with bio-fuel. Additionally, if Rowan University shuttles were to fuel up strictly at an on-campus bio-fuel station, it would create an easy opportunity for a return on investment.
	\\
	Going further, it would be easy to make using a greener alternative to fossil fuels popular if students and faculty could purchase discounted bio-fuel by presenting a RowanCard. This would enable another route for a return on investment.
	\\
	Overall, sustainably produced bio-fuel would enable Rowan University to reach its environmental goals quicker in more ways than one. Not only would transportation-based pollution be minimized, but food waste focused environmental damages would also be mitigated.
	
	\begin{table}[h!]
		\caption{Parking Lot Surface Areas}
		\begin{tabular}{|l|l|}
			\hline
			\textbf{Lot} & \textbf{Surface area (m\textsuperscript{2})} \\ \hline
			C            & 7,900                 \\ \hline
			D            & 7,000                 \\ \hline
			D-1          & 4,800                 \\ \hline
			O            & 8,000                 \\ \hline
			O-1          & 8,100                 \\ \hline
			B and B-1    & 15,600                \\ \hline
			J            & 6,200                 \\ \hline
			P and R      & 6,600                 \\ \hline
			Edgewood     & 7,900                 \\ \hline
			\textbf{Total} & 72,100				\\ \hline
		\end{tabular}
		\label{tbl:LotAreas}
	\end{table}

	
	

\bibliography{bib}
\bibliographystyle{apalike}
	
\end{document}          
