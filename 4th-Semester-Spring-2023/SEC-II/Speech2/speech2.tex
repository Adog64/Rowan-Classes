\documentclass[11pt]{article}
    
\title{Why I Stopped Using Microsoft Windows}
\author{Aidan Sharpe}
\usepackage{marginnote}
\reversemarginpar

\begin{document}
\maketitle

This is a personal story: a story of corruption, surveillance, deception, and greed. This is 
how one man monopolized operating systems.
\\
\\
The word computer used to describe an occupation. People, often women, were tasked with 
crunching numbers by hand. It was a slow and tedious job, but a very important one. Yet, 
computing goes back much further than this.
\\
\\
Our tale begins in ancient Babylon. Put yourself in the shoes of a grain merchant. 
The year is 1700 BC, and you are busy keeping track of trades and calculating how much 
customers owe. It is a challenging task at first look, but you have an abacus. Simply by 
sliding beads, addition, subtraction, multiplication, and division can be done with relative 
ease.
\\
\\
For hundreds of years, this was the extent of personal computing -- nothing but sliding beads. 
Progress was slow. The first analog computers were used for tracking astronomical events 
like predicting eclipses, and navigation by stars. The first general purpose computer was 
invented by Charles Babbage in 1833. It was the first device that resembled modern processors 
in function, and over the coming century, the computing industry would explode in complexity.
\\
\\
\marginnote{Pause for intensity.} Bill Gates was born on October 28th, 1955. His childhood was
filled with success (\textit{pause}) and computers. In 1973, Gates attended Harvard, but his
passion for writing computer software took over his life. He dropped out after two years to
move to Albuquerque, New Mexico at a company called MITS -- the company behind the first ever
personal computer. He was inspired by MITS to make his own company. Short for microcomputer 
software, he called it Micro-Soft. Yes, it was originally hyphenated.
\\
\\
Micro-Soft was an instant success. Their first product, a form of the once popular programming
language, BASIC, was a huge hit for hobbyists around the country. In 1981, Microsoft released
MS-DOS. A primitive operating system that functioned similarly to the command prompt found
in modern Windows. The first version of Windows hit the shelves in 1985 and functioned as a
graphical overlay for the strictly text-based DOS platform.
\\
\\
At this point in time, Windows was fairly bare bones. It had a file explorer, a notepad, a 
calculator, and a clock. The mid 1990s saw the first wave of internet popularity, 
and early web browsers like Netscape Navigator and Opera saw much success during this period. 
\\
\\
Things started to change when the now defunct Internet Explorer was included as a default
application. Since software had to be purchased at brick-and-mortar retail stores and installed on
three-and-a-half inch floppy diskettes, default apps were given a major advantage. This 
change caught the attention of antitrust lawyers.
\\
\\
\marginnote{Very noticeable pause} Recall the first thing I said, "This is a \textit{personal} speech."
So far, this has been nothing but other peoples' stories. So, where do \textit{I} come into this?
Well, (\textit{pause}) it is a little more than just me. Here is where \textit{we} come into it.
\\
\\
I am willing to bet that at least 90\% of you are running Microsoft Windows. For those who are,
the next minute or so is dedicated to \textit{aaaaall} you. While I'm talking I'd like all of
the Windows users to, and please try to do it \textit{silently}, boot up you laptop.
\\
\\
While logging in, some of you, and everyone using Windows 11, will be logging into a Microsoft
account. In newer versions of Windows,  the Microsoft account has nearly replaced the local
user.
\\
\\
Back to the antitrust suits. What the heck were they about? Well, and I'm going to spoil it,
Microsoft won. They got away with being able to bundle software into their operating system.
Now that all of you are logged into Windows, take a look at the \textit{Start} menu. Look at
all of those programs that you didn't install. Candy Crush? Bubble Witch? TikTok? 
\marginnote{Sound absurdly confused} What are those \textit{do-ing} there???
\\
\\
Bloat! Bloat, bloat, bloat. It's all bloat, folks. And it's \textit{NOT} at all necessary.
Now, some of these programs are useful. Some. Some of them take up space. And some of them,
usually the more well-hidden ones, are actually \textit{harmful}.
\\
\\
This was a problem almost 30 years ago, and it's still a problem today. 


\end{document}