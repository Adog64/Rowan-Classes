\documentclass[]{mla}

\usepackage{hyperref}

\title{Live Speaking Event Paper}
\professor{Professor Coughlan}
\author{Aidan Sharpe}
\course{Sophomore Engineering Clinic II}

\begin{document}
    
    When it comes to technical speaking, there are a lot of parts and pieces that must be taken into account to make an effective and clear point. 
    \\
    On March 9th, 2023, Christopher Contos gave a presentation on Life Cycle Assesments, a moderately technical topic. The following are some low-resolution details regarding how the presentation was organized. There were about 20 Sophomore Honors Engineers present in the audience along with a course instructor with a background in English and education. Contos guided the audience with a Google Slides Presentation to serve as a visual aid.
    \\
    Despite these low-resolution observations, to best convey my critique on Christopher Contos's technical presentation on \textit{Life Cycle Assessments for Sustainable Engineering} at a granular scale, I will be interlacing my review with commentary on concepts and that I find valuable in technical lectures and public speaking as a whole.
    \\
    When it comes to presenting an analysis tool such as Life Cycle Assessments (LCAs) or any technical topic there are several things that must be done to be able to make an effective point. The most important of these things are getting the audience engaged, keeping the audience involved, staying away from unnecessary tangents, and leaving the audience inspired or in awe. By incorporating these components, any public speaking presentation can be made effective.
    \\    
    The easiest way to get the audience engaged in a topic is by luring them in with a captivating introduction. Starting with the mindset that nobody in the audience knows or cares about the topic can be extremely beneficial. From my experience, whenever possible, have a captivating story prepared to really get the audience tuned in. The story only needs to be tangentially related to the main topic, but it does need to be something that keeps the audience in mind, so there should be nothing technical in the hook. After all, nobody in the audience cares about the topic yet.
    \\
    When it came to Contos's presentation, however, I certainly was not hooked in right away. I had already dealt with LCAs in the past, and suffice it to say, I was never really intrigued by them. That said, I \textit{was} ready and willing to learn more about them, but without a solid hook to lure me in, my attention was wandering from the start.
    \\
    However, the introduction is not the only part that must be captivating. Even if the hook is not very effective, it can be possible to win over the audience's attention. If done well, the audience will be both silent and alert throughout the presentation. Public speaking is a dialogue, and the audience will use body language to participate and tell the presenter how well he or she is doing. Therefore, it is important to make adjustments along the way.
    \\
    This ability to make adjustments was something that Contos did quite well. The presentation was obviously rehearsed, but it did not at all sound memorized. It almost had a bit of an impromptu aspect. Although his pacing was rather quick, he made sure to tailor his word choice to the audience to improve his conversational tone.
    \\
    On a similar note to audience engagement, one way to make sure the audience can follow the flow of ideas and logic is by avoiding unnecessary tangents. While some tangents may be fine and in some cases helpful, it is absolutely crucial that they are presented in such a way that they add to the delivery of the point and do not simply fill time. Some of the best examples of useful tangets can be found in videos on the YouTube channel, \textit{VSauce}. Michael Stevens, the host of the channel did a TEDx Talk almost a decade ago now, about how he tricks people into learning more through effective tangents.
    \\
    Similarly to word choice, Contos was also effective with his use of tangents. All of his tangents were related stories about his experience and served as examples above all. As far as tangents are concerned for this presentation, the correct amount were used and, importantly, used properly. Side information was kept to a minimum, which was probably the best choice for the topic of the presentation. Overall, I would not change this component of the delivery.
    \\
    The final portion of any presentation should be an effective and inspiring conclusion. The ending should tie it all together. To do so in the best way possible, three major parts must be in place.
    \\
    The first part to a great conclusion is a recap of the main points. This can be as simple as a restatement of the thesis statement, however it should be a more informed version now that the audience has hopefully learned a thing or two about the topic.
    \\
    The second part to a great conclusion is by making a powerful close to an extended metaphor. I am going to be completely honest, and I know that I am kind of breaking my narrarator-esque character here, but I am a real sucker for extended metaphors. They are an incredible way to keep your points tied together in a linear way, and after all, people usually learn best if they have an analogy and story-like structure to follow along with. My favorite extended metaphor of all time comes from, of all things, a video essay from a video game history YouTube channel called \textit{Ahoy}. The video, \textit{Nuclear Fruit: How the Cold War Shaped Video Games} ends with such an effective conclusion that not only do I watch his videos as soon as they release to this day, but I was also inspired to become a better writer.
    \\
    The third and final part to an effective conclusion is a big picture relation. This "wow" statement, if you will, can be anything from a call to action to a hopeful or inspiring outlook to a famous quote.
    \\
    Unfortunately, however, Contos did not include any parts of an effective conclusion. There were little to no concluding remarks at all. Rather, he jumped straight into taking questions. It was quite abrupt and frankly displeasing. I felt like I was lured to an edge of a cliff and left hanging. A conclusion can make or break a presentation, and in this case, the audience was left wanting more.
    \\
    While there were certainly some portions of the delivery that were completely missing, other than his pacing, the parts that were there were presented well and with great attention to the audience's needs. At the end of the day, I would be willing to see him talk again if he was given and took feedback. Christopher Contos has what it takes to be really good public speaker, especially considering his presentation demeanor. He was overall quite enjoyable to listen to, and he offered up valuable information.
    \\
    \vspace*{\fill}
    \noindent
    I have provided below links to both YouTube videos mentioned:
    \\
    \href{https://www.youtube.com/watch?v=15dxuAbTC0A}{Nuclear Fruit: How the Cold War Sharped Video Games}
    \\
    \href{https://youtu.be/u9hauSrihYQ}{Why do we ask questions? Michael "Vsauce" Stevens at TEDxVienna}
    \end{document}