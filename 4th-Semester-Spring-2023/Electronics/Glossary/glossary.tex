\documentclass{IEEEtran}

\usepackage{enumerate}

\title{Electronics Glossary}
\author{Aidan Sharpe}

\begin{document}
\maketitle

% ========== A ============
\IEEEPARstart{A}{lpha}
\begin{enumerate}
\item Active Region
  \subitem The area of a transistor I-V curve where current no longer increases with an increased voltage. Power is dissipating, quiescence achieved \\
\item Anode
  \subitem A negatively polarized electrode, pin, or terminal.\\
\item Astable Multivibrator
  \subitem A positive feedback device that utilizes hysteresis and transient response to generate a tuneable square wave from a DC input. \\

% ========== B ============
\IEEEPARstart{B}{ravo}\\
\item Bandpass Filter
  \subitem A type of electronic filter that allows a selected range of frequencies to pass through. \\
\item Bandreject Filter
  \subitem Like a flipped over bandpass filter. Only blocks a selected range of frequencies from passing through. \\
\item Bandwidth
  \subitem The size of a defined range of frequencies.
\item Base
  \subitem The terminal of a bipolar junction transistor that controls the open and closed switch behavior.\\
\item Biasing
\item Bipolar
  \subitem When + and - references are used at the rails. There is a common ground. 
\item Bipolar Junction Transistor
  \subitem A type of semiconducing transistor.\\
\item Biquad
  \subitem \\ 
\item Bode Plot
  \subitem A logarithmic scale graph showing gain vs. frequency. 
\item Brain Box
  \subitem A colloquial term for an Engine Control Module (ECM).\\ 
\item Buffer
  \subitem A type of circuit isolator, i.e unity gain follower\\
\item Bulk Capacitor
  \subitem \\ 
\item Bypass Capacitor
  \subitem A Bypass Capacitor is a capacitance that shorts AC signals to ground, so that any AC noise that may be present on a DC signal is removed, producing a much cleaner and pure DC signal. Usually about $0.1 \mu F$ \\
\ 

% ========== C ============
\IEEEPARstart{C}{harlie}\\
\item C
  \subitem The speed of light $3\times10^8 m/s$
\item Capacitor
  \subitem A passive transient linear device that stores energy in an electric field.\\
\item Cathode
  \subitem A positively polarize electrode, pin, or terminal\\
\item Center Tap Transformer
  \subitem A tranformer that with a central common terminal: offering a positive and negative voltage on either side with equal magnitude.\\
\item Clamp Diodes
  \subitem A diode that is used to force a voltage on the anode.\\
\item Collector
  \subitem The positive terminal of a BJT.\\
\item Common Mode Rejection Ratio (CMRR)
  \subitem $CMRR = 20\log_{10} \left(\frac{A_d}{A_{cm}}\right)$ \\
\item Cut-off
  \subitem The area of a transistor I-V curve where there is no current \\ 
\\

% ========== D ============
\IEEEPARstart{D}{elta}\\
\item DC Restorer
  \subitem \\
\item Denormalization
  \subitem \\ 
\item Dielectric
  \subitem A material that increases the affect of an electric field: often used to increase capacitance. Typically denoted by a $\kappa$ \\
\item Differential Amplifier
  \subitem \\
\item Diode
  \subitem A semiconducing device that allows current to travel in only one direction.\\
\item Distributed Parameter
  \subitem A component with properties along a length or area rather than localized at a point. Must be used to model a component when the component is not much smaller than one wavelength. \\ 
\item Drain
  \subitem The terminal 

% ========== E ============
\IEEEPARstart{E}{cho}\\
\item Electromagnetic Interference {EMI}
\subitem Also called radio-frequency interference (RFI) when in the radio frequency spectrum, is a disturbance generated by an external source that affects an electrical circuit by electromagnetic induction, electrostatic coupling, conduction, or radiation.\\ 
\item Electrolytic Capacitor
  \subitem A type of polarized capacitor that uses an electrolyte and oxide layer to increase the dielectric constant.\\
\item Electronic Design Automation (EDA)
  \subitem Specialized software used to design and simulate electronic devices\\
\item Emitter
  \subitem The negative terminal of a BJT.\\ 
\item Energy Assurance Plan (EAP)
  \subitem \\

% ========== F ============
\IEEEPARstart{F}{oxtrot}\\
\item Farad
  \subitem The base SI unit for capacitance. 
\item Filter
  \subitem A circuit that only allows certain frequencies through.
\item Forward Voltage
  \subitem The voltage at which a semiconducing device begins to conduct.\\
\item Fudge Factor
  \subitem An extra term added to an equation
\item Full Wave Rectifier
  \subitem A device that restricts the output voltage to one pole and inverts the sign of the opposite pole.\\

% ========== G ============
\IEEEPARstart{G}{olf}\\
\item Gain
  \subitem A logarithmic measure of amplification.\\
\item General Interconnect
\subitem A way \\ 
\item Giga-
  \subitem The metric prefix meaning one billion ($10^{9}$) times the base unit.\\

% ========== H ============
\IEEEPARstart{H}{otel}\\
\item Half Wave Rectifier
  \subitem A device that restricts the output voltage to one pole.\\
\item High side switch
  \subitem \\
\item Hysteresis
  \subitem \\

% ========== I ============
\IEEEPARstart{I}{ndia}\\
\item IGBTs
\item Interconnect
  \subitem \\
\item Interconnect Shielding
\subitem \\
\\

% ========== J ============
\IEEEPARstart{J}{uliet}\\
\\

% ========== K ============
\IEEEPARstart{K}{ilo}\\
\item Kelvin Leads
\subitem A clip, often a crocodile clip, that connects a force-and-sense pair to measure very low resistances using four-terminal sensing.\\ 
\item Kilo-
  \subitem The metric prefix meaning one thousand ($10^{3}$) times the base unit.\\

% ========== L ============
\IEEEPARstart{L}{ima}\\
\item Low pass filter
  \subitem A type of AC filter that eliminates high frequencies.\\
\item Low side switch
  \subitem A type of switch where the transistor is placed between the main circuit and ground.\\
\item Lumped Parameter
  \subitem \\ 
\\


% ========== M ============
\IEEEPARstart{M}{ike}\\
\item Mega-
  \subitem The metric prefix meaning one million ($10^{6}$) times the base unit.\\
\item Memristor
  \subitem \\
\item Metal Oxide Veristor
  \subitem \\
\item Micro-
  \subitem The metric prefix meaning one millionth ($10^{-6}$) of the base unit.\\
\item Milli-
  \subitem The metric prefix meaning one thousandth ($10^{-3}$) of the base unit.\\
\item MOSFET
  \subitem Metal Oxide Semiconducting Field Effect Transistor. Voltage ($V_{GS}$) controls the current ($I_D$). \\

% ========== N ============
\IEEEPARstart{N}{ovember}\\
\item Nano-
  \subitem The metric prefix meaning one billionth ($10^{-9}$) of the base unit.\\
\item Negative Feedback
  \subitem \\
\item Normalized response
  \subitem \\ 

% ========== O ============
\IEEEPARstart{O}{scar}\\
\item Omega ($\omega$)
  \subitem Symbol indicating angular frequency\\
\item Operational Amplifier (Op-Amp)
  \subitem An active device that enables isolation, comparison, and amplification.\\


% ========== P ============
\IEEEPARstart{P}{apa}\\
\item Pico-
  \subitem The metric prefix meaning one trillionth ($10^{-12}$) of the base unit.\\
\item Positive Feedback
  \subitem In the context of an operational amplifier, it implies some kind of connection from the output terminal to the non-inverting input. \\
\item Power Supply 
  \subitem A nonlinear circuit that can reliably supply a specific voltage or specific current. \\
\item Power Supply Distribution 
  \subitem \\

% ========== Q ============
\IEEEPARstart{Q}{uebec}\\\\
\item Quiescent
  \subitem When a device is dissapating power without a signal input.\\

% ========== R ============
\IEEEPARstart{R}{omeo}\\
\item Rectification
  \subitem Forcing polarity on a signal.\\
\item Resistor
  \subitem A purely resistive linear device with no transient properties.\\
\item Rheostat
  \subitem A type of variable resistor. Like a potentiometer without one of the end pins.\\

% ========== S ============
\IEEEPARstart{S}{ierra}\\
\item Sallen-Key LPF Circuit
  \subitem A 2-pole circuit with a non-inverting amplifier. \\                 
\item Saturation (for BJTs)
  \subitem The regime of operation for BJTs where increasing $V_{BE}$ increases $I_C$. \\
\item Saturation (for MOSFETs)
  \subitem The regime of operation for MOSFETs where $V_{GS}$ is positive and increasing ${V_DS}$ does not affect $I_D$ \\
\item Shielding
  \subitem A piece of metal, wrapped around a wire or electronic device used to minimize EMI noise and radiation.
\item Summing Amplifier
  \subitem A type of amplifier that adds voltages: can be inverting or non-inverting \\
\item Schottky Diode
  \subitem A low forward voltage, high switching speed, high reverse leakage current diode. \\

% ========== T ============
\IEEEPARstart{T}{ango}\\
\item Transistor
  \subitem A three terminal, active, semiconducing device that acts as an electronic switch.\\
\item Thermal Model Electronics
  \subitem A way to model the thermal behavior of electronic devices using circuit schematic symbols.\\
\item Transformer
  \subitem An inducting device that can step up or step down AC voltage while having little power loss. \\
\item Triaxial Cable
  \subitem A cable with three concentric conductors used to minimize EMI. \\ 
\item Triode
  \subitem The operating regime of a MOSFET where increasing $V_{DS}$ increases $I_D$.\\
\item Twisted pair
\subitem Two wires wrapped around one another to minimize loop area, thereby decreasing EMI. \\

% ========== U ============
\IEEEPARstart{U}{niform}\\
\item Unipolar
  \subitem Simply a voltage and ground as references. As opposed to Bipolar with + and - reference voltages. \\ 

% ========== V ============
\IEEEPARstart{V}{ictor}\\
\item V-I Response
  \subitem A visual comparison of the voltage and current response of an electronic device.\\
\item Virtual Ground
  \subitem A voltage that is very close to ground caused by amplifier feedback. \\

% ========== W ============
\IEEEPARstart{W}{hiskey}\\
\\

% ========== X ============
\IEEEPARstart{X}{-ray}\\
\\

% ========== Y ============
\IEEEPARstart{Y}{ankee}\\
\\

% ========== Z ============
\IEEEPARstart{Z}{ulu}\\
\item Zener Diode
  \subitem A special type of diode with a set reverse breakdown voltage.\\
\item Zener Regulator
  \subitem A type of power regulator that clamps voltage using a Zener Diode.\\
\end{enumerate}
\end{document}